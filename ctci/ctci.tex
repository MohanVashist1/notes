\documentclass[12pt]{article}

\usepackage{upgreek}

\usepackage{amsmath}

\usepackage{graphicx}
\graphicspath{ {imgs/} }

\usepackage{dsfont}

\usepackage{mathtools}

\usepackage{hyperref}

\usepackage[utf8]{inputenc}

\usepackage{mathtools}

\usepackage{textcomp}

\usepackage[english]{babel}

\usepackage{tikz}

\usepackage{tcolorbox}

\usepackage{amsthm,amssymb}

\setlength{\parindent}{0cm}

\renewcommand\qedsymbol{$\blacksquare$}

\usepackage{fancyhdr}
 
\pagestyle{fancy}
\fancyhf{}
\fancyhead[LE,RO]{Cracking the Coding Interview}
\fancyhead[RE,LO]{Joshua Concon}
\fancyfoot[CE,CO]{\leftmark}
\fancyfoot[LE,RO]{\thepage}


\begin{document}

\title{Cracking the Coding Interview Notes}
\date{Gayle Laakmann Mcdowell}
\author{Joshua Concon}
\maketitle
If you find any problems in these notes, feel free to contact me at conconjoshua@gmail.com.

\tableofcontents

\pagebreak

\section{Chapter 6}

\subsection{Big $O \slash \Theta \slash \Omega$ }

\paragraph{Big O}{Upper bound on the time of the algorithm}
\paragraph{Big \Omega}{Lower bound on time of algorithm}
\paragraph{Big \Theta}{both Big $\Omega$ and Big $O$}

Note: The industry definition of "Big O" is closer to what academics mean by Big $\Theta$

\subsection{Best, Worst, Expected Cases}

We can describe an algorithm's run time in 3 different ways:
\begin{enumerate}
    \item Best Case (This is rarely helpful)
    \item Worst Case (This is usually the most helpful) (Is usually the "unlucky" case)
    \item Excepted Case (Useful for more probability dependent algorithms)
\end{enumerate}

\subsection{Space Complexity}

Sometimes we care about the amount of memory or space an algorithm requires.
\\
\\
Note: Drop constants and other non-dominant terms for Big $O \slash \Theta$

\subsection{Amortized Time}

Considering multiple operations all as one operation, and finding the run time of the aggregated operation. 
\subsection{Recursive Runtime}

f


\end{document}