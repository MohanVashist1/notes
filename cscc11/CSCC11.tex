\documentclass[12pt]{article}

\usepackage{upgreek}

\usepackage{amsmath}

\usepackage{graphicx}
\graphicspath{ {imgs/} }

\usepackage{dsfont}

\usepackage{mathtools}

\usepackage{hyperref}

\usepackage[utf8]{inputenc}

\usepackage{mathtools}

\usepackage{textcomp}

\usepackage[english]{babel}

\usepackage{tikz}

\usepackage{tcolorbox}

\usepackage{amsthm,amssymb}

\setlength{\parindent}{0cm}

\renewcommand\qedsymbol{$\blacksquare$}

\usepackage{fancyhdr}
 
\pagestyle{fancy}
\fancyhf{}
\fancyhead[LE,RO]{Machine Learning -- Winter 2020}
\fancyhead[RE,LO]{Joshua Concon}
\fancyfoot[CE,CO]{\leftmark}
\fancyfoot[LE,RO]{\thepage}


\begin{document}

\title{CSCC11: Machine Learning\\ Lecture Notes}
\date{University of Toronto Scarborough -- Winter 2020}
\author{Joshua Concon}
\maketitle
Instructor is Dr. David Fleet. If you find any problems in these notes, feel free to contact me at conconjoshua@gmail.com.

\tableofcontents

\pagebreak

\section{LEC 1: Tuesday, January 7, 2020}

\subsection{Machine Learning Definitions}

\paragraph{AI View} Automatic Learning
\paragraph{Software Engineering View} Can be fine-tuned
\paragraph{Statistics View} Machine Learning is fast Statistics

\subsection*{Machine Learning methods are broken into 2 phases}
\paragraph{Training} a model is learned from a collection of training data
\paragraph{Application} the model is used to make decisions about some new test data

\subsection*{Types of Machine Learning}
\begin{itemize}
  \item Supervised Learning: The training data is labelled with the correct answers
  \begin{itemize}
    \item Classification: Outputs are discrete labels
    \item Regression: Outputs are real-valued
  \end{itemize}
  \item Unsupervized Learning: Deriving patterns and structure from unlabelled data
  \item Reinforcement Learning: An agent seeks to learn optimal actions to take based on state of the world, and learn from the consequences of past actions
\end{itemize}

\subsection{A Simple Problem}
Consider a problem where the goal is to fit a 1D curve. There are many curves that may fit.
\\
\\
Using Machine Learning requires us to make certain choices:

\begin{enumerate}
  \item How do we parameterize the curve? (linear, quadratic, sine...)
  \item What criteria (objective function) do we use to judge the quality of the fit?
  \item How long are we willing to wait for a solution (or can we use approximations)?
  \item How do we prevent overfitting?
\end{enumerate}

\textbf{Overfitting} is when a model fits the training data well, but performs poorly on test data.

Properties of the input that are not relevant to the task at hand are called \textbf{Noise}.


\end{document}