\documentclass[12pt]{article}

\usepackage{upgreek}

\usepackage{amsmath}

\usepackage{amsthm}

\usepackage{graphicx}
\graphicspath{ {imgs/} }

\usepackage{dsfont}

\usepackage{mathtools}

\usepackage{hyperref}

\usepackage[utf8]{inputenc}

\usepackage{mathtools}

\usepackage{textcomp}

\usepackage[english]{babel}

\usepackage{tikz}

\usepackage{tcolorbox}

\usepackage{amsthm,amssymb}

\setlength{\parindent}{0cm}

\renewcommand\qedsymbol{$\blacksquare$}

\usepackage{fancyhdr}
 
\pagestyle{fancy}
\fancyhf{}
\fancyhead[LE,RO]{Introduction to Analysis -- Winter 2018}
\fancyhead[RE,LO]{Joshua Concon}
\fancyfoot[CE,CO]{\leftmark}
\fancyfoot[LE,RO]{\thepage}

\theoremstyle{plain}

\newtheorem*{remark}{Remark}
\newtheorem*{note}{Note}
\newtheorem{theorem}{Theorem}[section]

\theoremstyle{definition}

\newtheorem{definition}[theorem]{Definition}
\newtheorem{lemma}[theorem]{Lemma}
\newtheorem{example}[theorem]{Example}
\newtheorem{proposition}[theorem]{Proposition}
\newtheorem{corollary}[theorem]{Corollary}
\newtheorem{property}[theorem]{Property}

\begin{document}

\title{MATB43: Introduction to Analysis\\ Lecture Notes}
\date{University of Toronto Scarborough -- Winter 2018}
\author{Joshua Concon}
\maketitle
Pre-reqs are MATA37. Instructor is John Scherk. If you find any problems in these notes, feel free to contact me at conconjoshua@gmail.com.

\tableofcontents

\pagebreak

\section{Friday, January 5, 2018}

\subsection{Countability}

Countability is all about the cardinality of sets

\begin{definition}
	For sets $X,Y$, they have the same cardinality if there exists a bijection between them ($f:X\mapsto Y$ has an inverse as well). Equality of cardinality of 2 sets is written as $card(X) = card(Y)$
\end{definition}

\begin{example}
	$card(\{ 1,2,3,4,5 \}) = card(\{ a,b,c,d,e \}) $
\end{example}
\begin{example}
	$card(\mathbb{N}) = card(2\mathbb{N})$
\end{example}

This example is correct because $f:\mathbb{N} \mapsto 2\mathbb{N}, f(n)=2n$ is a bijection.

\begin{example}
	$card(\{\text{set of odd numbers}\}) = card(\mathbb{N})$
\end{example}

This example is correct because $g:\mathbb{N} \mapsto \{\text{set of odd numbers}\}\\ g(n)=2n+1, n\in\mathbb{N}$

\begin{definition}
	$X$ is \textbf{finite} if for some $n\in\mathbb{N}$ there exists a bijection where $A=\{ 1,2,...,n \}$ $f:A\mapsto X$. So essentially $card(A) = card(X)$
\end{definition}

Intuitively, we can label the elements of $X$ as $X = \{ x_1, x_2, ..., x_n \}$

\begin{definition}
	$X$ is \textbf{inifinite} if $X$ is not finite.
\end{definition}

Examples of this include: $\mathbb{N}, 2\mathbb{N}, \mathbb{Q}, \mathbb{R}$

However, not all infinite sets have the same cardinality.

\begin{definition}
	$X$ is \textbf{countable} if $card(X) = card(A)$ where $A \subseteq \mathbb{N}$
\end{definition}

Again intuitively, we can label the elements of $X$ as $X = \{ x_1, x_2, ..., x_n, ... \}$ (using elements of $\mathbb{N}$)

\begin{example}
	$2\mathbb{N}$
\end{example}

\begin{example}
	The set of odd numbers
\end{example}

\begin{example}
	$\mathbb{Z}$
\end{example}

\begin{theorem}
	In general, if $Z=X \cup Y$ and $Y,X$ are both countable then $Z$ is countable as well
\end{theorem}

\begin{proof}
	label elements of $X = \{ x_1, x_2, ..., x_n, ... \}$\\
	label elements of $Y = \{ y_1, y_2, ..., y_n, ... \}$\\
	define $h:\mathbb{N}\mapsto \mathbb{Z}$ as, for $n\in\mathbb{N}$:\\
	$$h(2n-1)=x_n$$
	$$h(2n)=y_n$$
	Then $h$ is a bijection, therefore $Z$ is also countable
\end{proof}

\begin{example}
	$\mathbb{N} \times  \mathbb{N}$
\end{example}

This example is countable as we can label its elements in the following pattern.

$$(1,1) \mapsto 1$$
$$(2,1) \mapsto 2$$
$$(1,2) \mapsto 3$$
$$(1,3) \mapsto 4$$
$$(2,2) \mapsto 5$$
$$(3,1) \mapsto 6$$
$$(4,1) \mapsto 7$$

And so on, in this pattern. Intuitively, if you list out all the pairs of $\mathbb{N} \times  \mathbb{N}$ like a matrix, this would create a sort of zig zag pattern.

\begin{proposition}
	Suppose $X$ is countable and $Y \subset X$, then $Y$ is either finite (which also means that it is countable) or just countable
\end{proposition}

\begin{proof}
	write $X = \{ x_1, x_2, ..., x_n, ... \}$\\
	let $j_1$ be the smallest index such that $x_{j_1} \in Y$\\
	let $j_2$ be the smallest index such that $x_{j_2} \in Y, j_2 > j_1$\\
	let $j_3$ be the smallest index such that $x_{j_3} \in Y, j_3 > j_2$\\
	...\\
	and so on.\\
	\\
	Now either:\\
	\textbf{This process terminates}, which means that $Y = \{ x_{j_1} , ..., x_{j_n} \}$ for some $j_1, ..., j_n$ which $Y$ is finite.\\
	\\
	or:\\
	\textbf{Y is countable}, since $Y = \{ x_{j_1}, x_{j_2} ,..., x_{j_n}, ... \}$
\end{proof}

\begin{proposition}
	$\mathbb{Q}$ is countable
\end{proposition}

\begin{proof}
	Note that $\mathbb{Q} = \mathbb{Q}^- \cup \{ 0 \} \cup \mathbb{Q}^+$\\
	so we must check that $\mathbb{Q}^+$ is countable, note that
	$$\mathbb{Q}^+ = \{ \frac{m}{n} | m,n \in \mathbb{N}, n \neq 0 \}$$
	
	Since it is a pairing of two natural numbers $(m,n)$, this is a subset of $\mathbb{N} \times \mathbb{N}$, which means that $\mathbb{Q}^+$ is countable, we can say that same thing for $\mathbb{Q}^-$ as well, note that it is
	$$\mathbb{Q}^+ = \{ \frac{-m}{n} | m,n \in \mathbb{N}, n \neq 0 \}$$
	and since $\mathbb{Q}^+$ and $\mathbb{Q}^-$ are both countable and $\{ 0 \}$ is a set of cardinality 1, therefore $\mathbb{Q}$ is countable
\end{proof}

\begin{theorem}
	Let $S = \{ s=(s_1, s_2, ...) | s_j = 0 \text{ or } 1 \forall j \}$ (note here that $s$ is a sequence). Then $S$ is not countable
\end{theorem}

\begin{proof}
	For proof by contradiction, suppose $S$ is countable.\\
	We can then label the elements of $S$ as $S^1, S^2, ... , S^n, ... \in S$.\\
	\textbf{So an example of this would be:}\\
	$$S^1 = (s^1_1, s^1_2, s^1_3, ..., s^1_n, ...)$$
	$$S^2 = (s^2_1, s^2_2, s^2_3, ..., s^2_n, ...)$$
	$$...$$
	$$S^m = (s^m_1, s^m_2, s^m_3, ..., s^m_n, ...)$$
	$$...$$
	And we will define a $t\in S$ as follows, let $t= (t_1, t_2,..., t_m, ...)$\\
	$$t_1 = \begin{cases}0, &\text{if $S^1_1=1$}\\
				1, &\text{if $S^1_1=0$}
				\end{cases}$$
	$$t_2 = \begin{cases}0, &\text{if $S^2_2=1$}\\
				1, &\text{if $S^2_2=0$}
				\end{cases}$$
	$$...$$
	$$t_m = \begin{cases}0, &\text{if $S^m_m=1$}\\
				1, &\text{if $S^m_m=0$}
				\end{cases}$$
	$$...$$
	Therefore $t \neq S^m, \forall m$. Therefore this is a contradiction as $t$ was not in the listed elements of $S$, therefore $S$ is not countable.
\end{proof}

\begin{corollary}
	$\mathbb{R}$ is not countable
\end{corollary}

\begin{proof}
	Regard $\mathbb{R}$ as a set of infinite decimal fractions.\\
	Identify $s\in S$ with the decimal number $0.s_1 s_2 ...s_n ...$ so $S$ can be regarded as a subset of $\mathbb{R}$. If $\mathbb{R}$ were countable, then $S$ would be countable, therefore $\mathbb{R}$ must not be countable.
\end{proof}

\subsection{Real Numbers}

Examples of real numbers that are not rational include: $\pi, e$ (which are algebraic numbers), $\sqrt{2}, \sqrt{3}, \frac{\sqrt{5}+1}{2}$ (which are roots of polynomial equations with integer coefficients.)

\subsection{Algebraic Numbers}

The set of Algebraic Numbers is a set $\overline{\mathbb{Q}}$ such that
$$\overline{\mathbb{Q}} = \{ x\in \mathbb{R} | x^n+ a_{n-1} x^{n-1} + ... + a_0 = 0 \text{ for some } a_0,...,a_{n-1} \in \mathbb{Z} \}$$

Note that $\mathbb{Q} \subset \overline{\mathbb{Q}}$, and in fact, $\overline{\mathbb{Q}}$ is countable.\\
\\
if $x \in (\mathbb{R} \setminus \overline{\mathbb{Q}})$ then. $x$ is transcendental.

\subsection{Bounds}

\begin{definition}
	$X \subset \mathbb{R}, X$ is \textbf{bounded above} if $\exists a \in \mathbb{R}$ such that $a \geq x, \forall x \in X$
\end{definition}

\begin{definition}
	$X \subset \mathbb{R}, X$ is \textbf{bounded below} if $\exists a \in \mathbb{R}$ such that $a \leq x, \forall x \in X$
\end{definition}

\begin{example}
	$X = \{ x\in\mathbb{R} | x^2 \leq 2 \}$ is bounded above by $\frac{3}{2}$ since $(\frac{3}{2})^2 \geq 2$
\end{example}

\begin{definition}
$a$ is the \textbf{least upper bound} of $X$ if $a$ is an upper bound of $X$ and if $b<a$, then there exists $x\in X$ such that $x > b$.
\end{definition}

We write $a = lub(X)$ or $a=sup(X)$ if $a$ is the least upper bound of $X$

\begin{definition}
$a$ is the \textbf{greatest lower bound} of $X$ if $a$ is an lower bound of $X$ and if $b>a$, then there exists $x\in X$ such that $x < b$.
\end{definition}

\begin{example}
	if $X= \{ x | x^2 \leq 2 \}$, then $sup(X)=\sqrt{2} \not\in \mathbb{Q}$
\end{example}

\begin{property}
	if $X \subset \mathbb{R}$ is bounded above then there exists $a\in\mathbb{R}$, a least upper bound of X
\end{property}

\begin{theorem}
	given $a,b\in\mathbb{R}$ $a,b>0$ $\exists n \in \mathbb{N}$ such that $na > b$ (This is known as the \textbf{archemedian property})
\end{theorem}

\begin{proof}
	Suppose that $\forall n \in \mathbb{N}$, $na \leq b$ implies that $b$ is an upper bound for $X = \{ na | n\in\mathbb{N} \}$.\\
	\\
	Since $X$ is bounded above, let $c=sup(X)$, this implies that $c-a$ is not an upper bound for $X$, which further implies that there exists an $a\in\mathbb{N}$ such that $na > c-a$, and that
	\begin{align*}
		na &> c-a\\
		na+a &> c\\
		(n+1)a &> c\\
		c < (n+1)a \in X
	\end{align*}
	
	This is impossible since we've previously stated that $c$ is the upper bound. Therefore $X$ is not bounded above.
\end{proof}

\begin{theorem}
	given $c,d\in\mathbb{R}$, there exists $q = \frac{m}{n} \in\mathbb{Q}$ such that $c< q <d$
\end{theorem}

\begin{proof}
	We want $c < \frac{m}{n} < d $ iff $nc < m < nd$.\\
	let $\epsilon = d-c > 0$\\
	pick $n\in\mathbb{N}$ such that $\frac{1}{n} < \epsilon$\\
	pick $m$, such that $m > nc$\\
	(since we also need $m<nd$, choose $m$ to be as small as possible such that $m-1 \leq nc < m$)
	
	\begin{align*}
		m-1 &\leq nc \leq m\\
		\frac{m}{n} - \frac{1}{n} = \frac{m-1}{n} &\leq c \leq \frac{m}{n}
	\end{align*}
	 or
	 $$\frac{m}{n} \leq c+\frac{1}{n} < c + \epsilon =d$$
	 So now we have $c < \frac{m}{n} < d$ as desired.

\end{proof}

\newpage

\section{Monday, January 8, 2017}

\subsection{Sequences - Review}

This section will be just review of MATA37 and will be concerning sequences of the real numbers $\{ a_n \}$.

\begin{example}
    $\{ \frac{1}{n} \} = 1,\frac{1}{2}, \frac{1}{3}, ..., $
\end{example}

\begin{example}
    $\{ \frac{(-1)^n}{n} \} = -1,\frac{1}{2}, \frac{-1}{3}, ..., $ This sequence oscillates back and forth.
\end{example}

\begin{example}
    $\{ (-1)^n \} = -1,1,-1,1, ..., $ This sequence oscillates back and forth aswell.
\end{example}

\begin{example}
    $\{ x_1, x_2, ...  \}$ where this sequence enumerates $\mathbb{Q}$. This sequence 'bounces around wildy'.
\end{example}

\begin{definition}
    $x\in\mathbb{R}$ is the \textbf{limit of a sequence $\{x_n \}$} (so that in converges to this number)\\
    $\lim_{n\to\infty} x_n = a$, if given some tolerance $\epsilon > 0$.\\
    There exists $N$, such that $n \geq N$ will lie in the interval $a-\epsilon, a+\epsilon$, aka $|x_n - a| < \epsilon$.
\end{definition}

\begin{example}
    $\lim_{n\to\infty} \frac{1}{n} = 0$
\end{example}

\begin{example}
    $\lim_{n\to\infty} \frac{(-1)^n}{n} = 0$
\end{example}

\begin{example}
    example 2.4 and 2.3 both have no limit
\end{example}

\begin{proposition}
    If a sequence $\{x_n \}$ has a limit, then this limit is unique, and the sequence is bounded.
\end{proposition}

\begin{proposition}
    Supposed that $\{x_n \} , \{y_n \} \subset \mathbb{R} $ are convergent sequences, so that $\lim_{n\to\infty} x_n = a$, $\lim_{n\to\infty} y_n = b$ then:
    \begin{enumerate}
        \item $\{x_n + y_n \}$ converges and $\lim_{n\to\infty} (x_n + y_n) = a+b$
        \item $\{x_n y_n \}$ converges and $\lim_{n\to\infty} (x_n y_n) = ab$
        \item if $y_n \neq 0, \forall n$, and $b\neq 0$ then $\{\frac{x_n}{y_n} \}$ converges and $\lim_{n\to\infty} ( \frac{x_n}{y_n}) = \frac{a}{b}$
    \end{enumerate}
\end{proposition}

\begin{definition}
    A sequence is \textbf{monotone} if it is either increasing, or decreasing.
\end{definition}

\begin{proposition}
    A bounded monotone sequence converges.
\end{proposition}

\begin{proof}
    Suppose that $\{x_n \}$ is a bounded increasing (monotone) sequence, such that
    $$x_1 \leq x_2 \leq x_3 \leq ...$$
    And there exists $A\in\mathbb{R}$ such that $x_n \leq A, \forall n$\\
    Therefore, there exists a least upper bound $a \leq A$.\\
    \\
    \underline{claim:} $x_n \to a$ as $n \to \infty$\\
    \\
    take $\epsilon > 0$, since $a$ is the least upper bound of $\{ x_n \}$, there exists $N$ such that $a-\epsilon < x_N \leq a$. But $\{ x_n \}$ is increasing. Therefore $\forall n \geq N, a-\epsilon < x_N \leq x_n \leq a$. This implies that $\lim_{n\to\infty} x_n = a$
\end{proof}

\newpage

\section{Friday, January 12, 2017}

\subsection{Monotone Sequences}

\underline{i.e.} Sequences which are increasing or decreasing

\begin{proposition}
    A bounded monotone sequence converges.
\end{proposition}

\begin{proof}
    For an increasing sequence $\{ x_n \}, \lim_{n\to\infty} x_n = sup(\{ x_n \})$, suppose $\{ x_n \}$ is decreasing and bounded below, then $\{ -x_n \}$ is increasing and bounded above. This implies that $\lim_{n\to\infty} -x_n$ and $\lim_{n\to\infty} x_n$ both exist, and $$\lim_{n\to\infty} x_n = \lim_{n\to\infty} -x_n$$
\end{proof}

So now we're going to prove that Every sequence has a monotone subsequence along with other propositions on bounded monotone sequences. This will allow us to prove the Bolzano-Weierstrass Theorem, which states that every bounded sequence has a convergent subsequence, and this will help us prove the Cauchy property for sequences.

\begin{example}
    Lets look at the following sequence:
    $$x_n = 1 + \frac{1}{2} + ... + \frac{1}{n} - logn, n\geq 1$$
    So we want to show that this converges, we'll show that if we show that $\{ x_n \}$ is decreasing and is bounded below by 0.\\
    \\
    The limit ($\lim_{n\to\infty} x_n$) is actually a mysterious number that we don't know too much about.
    
    \begin{tcolorbox}
    \underline{Recall:} that $logn = \int^n_1 \frac{dt}{t}$\\
    So if we consider the space under the graph of $\frac{1}{t}$ from $n$ to $n+1$, we get the following inequality
    
    $$\frac{1}{n+1} < log(n+1) - logn < \frac{1}{n}$$
    Which implies that
    $$\frac{1}{n+1} - log(n+1) + logn < 0$$
    
    \end{tcolorbox}
    Now consider the following for an arbitrary $n$
    $$x_{n+1} = x_n + (\frac{1}{n+1} + logn - log(n+1))$$
    but since $\frac{1}{n+1} - log(n+1) + logn < 0$, this would mean that $x_{n+1} < x_n$, so this sequence is decreasing.\\
    \\
    Now consider the following inequality derived from the recall block:
    $$\sum^m_{n=1}(log(n+1) - logn) < \sum^m_{n=1}\frac{1}{n}$$
    The left side of this inequality telescopes, giving us
    $$log(m) < log(m+1) < 1 + \frac{1}{2} + ... + \frac{1}{m}$$
    And since the left most side is greater than the right most side, that means that $x_m > 0, \forall m$.
    
\end{example}

\subsection{Subsequences}

\begin{definition}
    Let $\{ x_i \}$ be a sequence of the real numbers, pick a finite set of indices $j_1 < j_2 < ... < j_n < ....$.\\
    A \textbf{Subsequence} of $\{ x_i \}$ is $\{ x_{j_1}, x_{j_2}, ..., x_{j_n},... \}$
\end{definition}

\begin{example}
    Considering the sequence $\{ x_n \} = \{ 1,\frac{1}{2}, \frac{1}{3}, ..., \frac{1}{n}, ... \}$
    
    \begin{enumerate}
        \item $\{ \frac{1}{2n} \}$ is a subsequence of $\{ x_n \}$.
        \item $\{ \frac{1}{2^n} \}$ is a subsequence of $\{ x_n \}$.
        \item $\{ (-1)^n \}$ is also a subsequence of $\{ x_n \}$, note that it does not converge, but has convergent subsequences of $\{ (-1)^{2n} \}$ and $\{ (-1)^{2n+1} \}$
    \end{enumerate}
\end{example}

\begin{definition}
    Call a term $x_m$ \textbf{dominant} if $x_n \leq x_m, \forall n \geq m$
\end{definition}

\begin{proposition}
    Every real number sequence has a monotone subsequence
\end{proposition}

\begin{proof}
    There are 2 cases:\\
    \underline{Case 1: infinitely many dominant terms}\\
    let $\{ x_{j_1}, x_{j_2}, ..., x_{j_n},... \}$ be the sequence of dominant terms. By definition, $x_{j_1} \geq x_{j_2} \geq ... \geq x_{j_n} \geq...$\\
    so $\{ x_{j_n} \}$ is a decreasing sequence, which is monotone.
    \\
    \\
    \underline{Case 2: only finitely many dominant terms}\\
    Pick an index $j$, so that $x_{j_1}$ is the first term beyond all dominant terms in the sequence $(\exists i, i<j_1, x_i \text{ is the last dominant term})$.\\
    Since $x_{j_1}$ is not dominant, then $\exists j_2 > j_1$ where $x_{j_2} > x_{j_1}$\\
    Since $x_{j_2}$ is not dominant, then $\exists j_3 > j_2$ where $x_{j_3} > x_{j_2}$\\
    ...\\
    and so on\\
    By induction, we construct an increasing subsequence $\{ x_{j_m} \}$ of $\{ x_n \}$.\\
    \\
    By these 2 cases, every real number has a monotone sequence.
    
\end{proof}

\begin{theorem}
    (Bolzano-Weierstrass Theorem) every bounded sequence has a convergent subsequence
\end{theorem}

\begin{proof}
    Given a bounded sequence $\{ x_n \}$ of the real numbers, there exists a monotone subsequence $\{ x_{j_n} \}$.\\
    The monotone subsequence $\{ x_{j_n} \}$ is also bounded since $\{ x_n \}$ is bounded.\\
    This implies that $\lim_{n\to\infty} x_{j_n} $ exists.
\end{proof}

\begin{definition}
    \textbf{Cauchy Property:} intuitively, in a convergent sequence, the terms get closer and closer as $n\to\infty$. More precisely:\\
    $\{ x_n \}$ is a real number sequence, given $\epsilon > 0, \exists N$ such that for $m,n > N, |x_m - x_n|<\epsilon$.
\end{definition}

\begin{proposition}
    Suppose that $\{ x_n \}$ converges to $a$, then $\{ x_n \}$ satisfies the Cauchy property.
\end{proposition}

\begin{proof}
    given $\epsilon > 0$, then $\exists N$ such that $\forall n > N, |x_n - a| < \frac{\epsilon}{2}$, so if $m > N$ then $|x_m - a| < \frac{\epsilon}{2}$ as well.\\
    This implies that
    $$|x_m - x_n| = |x_m -a + a - x_n| \leq |x_m -a| + |x_n-a| < \frac{\epsilon}{2} + \frac{\epsilon}{2} = \epsilon$$
\end{proof}

So now we want to show that if a sequence satisfies the Cauchy property, then it converges. We'll do this by first showing that a bounded sequence with the Cauchy property implies that sequence has a convergent subsequence. The second and final step is to show that the limit of a convergent subsequence is the limit of the original sequence.\\
\\
\begin{proposition}
    If $\{ x_n \}$ satisfies the Cauchy property, then it's bounded.
\end{proposition}

\begin{proof}
    By definition, there exists $N$ so that for $m,n > N, |x_m - x_n| <1$ in particular, $\forall n > N, |x_n - x_{N+1}| < 1$, which gives us:
    \begin{align*}
        -1 <& x_n - x_{N+1} < 1\\
        x_{N+1}-1 <& x_n < x_{N+1}+1
    \end{align*}
    
    \underline{Note:} $x_{N+1}$ is fixed (a constant)\\
    \\
    Let $A = min\{ x_1, .... ,x_N, x_{N+1}-1 \}$\\
    Let $B = max\{ x_1, .... ,x_N, x_{N+1}-1 \}$\\
    This implies that for all $n$, $A \leq x_n \leq B$, therefore Bounded.
    
\end{proof}

\begin{proposition}
    A sequence with the Cauchy property is convergent.
\end{proposition}

\begin{proof}
    Let $\{ x_n \}$ be a sequence with the Cauchy property.\\
    Since $\{ x_n \}$ is bounded, by the Bolzano--Weiestrass Theorem, there exists a convergent subsequence $\{ x_{j_m} \}$. Now let:
    $$a = \lim_{m\to\infty} x_{j_m}$$
    given $\epsilon, \exists M$ so that for all $m > M, |x_{j_m} - a| < \frac{\epsilon}{2}$.\\
    \\
    \underline{Note that $j_m \geq m$}\\
    \\
    The Cauchy propert implies that $\exists N$ so that for all $m,n > N, |x_m - x_n| < \epsilon$.\\
    \\
    We'll pick $P = max(N,M)$, then for $m,n > P$ we have:
    
    \begin{align*}
        |x_n - a| &= |x_n + x_{j_m} - x_{j_m} - a|\\
        &= |x_n - x_{j_m}| + |x_{j_m} - a|\\
        &< \frac{\epsilon}{2} + \frac{\epsilon}{2}\\
        &= \epsilon
    \end{align*}
\end{proof}

\begin{example}
    $$x_n = \sum^n_{k=1} \frac{1}{k^2}$$
    Verify the Cauchy property.\\
    \\
    Take $m>n, x_m - x_n = \sum^m_{k-n+1} \frac{1}{k^2}$. Now $\frac{1}{k} < \frac{1}{k(k-1)} = \frac{1}{k-1} - \frac{1}{k}$. This implies that
    \begin{align*}
        &\implies \sum^m_{k-n+1} \frac{1}{k^2} < \sum^m_{k-n+1} (\frac{1}{k-1} - \frac{1}{k})\\
        &\overset{\text{telescoping}}{\implies} \sum^m_{k-n+1} \frac{1}{k^2} < \frac{1}{m} - \frac{1}{n} < \frac{1}{m}\\
        &\implies (x_m - x_n) < \frac{1}{m}\\
    \end{align*}
    This implies that $x_n$ is convergent, in fact
    $$\lim_{n\to\infty} x_n = \sum^\infty_{k=1} \frac{1}{k^2} = \frac{\pi^2}{6}$$
\end{example}

\newpage

\section{Monday, January 15, 2017}

\subsection{Cauchy}

\begin{remark}
	Let $\{ x_n \}$ be a sequence of real numbers. Given $\epsilon > 0$, there exists $N$ such that for $m,n > N, |x_m- x_n|<\epsilon$. If $\{ x_n \}$ converges then $\{ x_n \}$ satisfies the cauchy property.

\end{remark}

The converse also holds. So if $\{ x_n \}$ satisfies the cauchy property, then it converges. But why? It holds because:
\begin{enumerate}
	\item $\{ x_n \}$ must be bounded
	\item Therefore it has a convergent subsequence $\{ x_{j_m} \}$, assume  $\{ x_{j_m} \}$ converges to $a$.
	\item Then $\{ x_n \}$ converges to $a$ as $n\to\infty$
\end{enumerate}

\subsection{Series}

This section will be focused on 
\begin{definition}
$$\sum^\infty_{n=0} a_n$$ which is an \textbf{infinite sum} of real numbers
\end{definition}

\begin{definition}
	The following is a \textbf{partial sum}:
	$$S_n = \sum^n_{k=0} a_k$$
\end{definition}

\begin{definition}
	The series $\sum^\infty_{n=0} a_n$ converges if $\lim_{n\to\infty} S_n$ exists. If $\lim_{n\to\infty} S_n = a$, then we write $a = \sum^\infty_{n=0} a_n$  as the \textbf{Sum of the series}. However, if $\lim_{n\to\infty} S_n$ does not exist, then the series diverges.
\end{definition}

\begin{example}
	The geometric series $\sum^\infty_{n=0} a^n$ converges if $|a|<1$ and diverges otherwise. If the series does converge, it converges to:
	$$\sum^\infty_{n=0} a^n = \frac{1}{1-n}$$
\end{example}

\begin{property}
	If $\sum^\infty_{n=0} a_n$, $\sum^\infty_{n=0} b_n$ both converge, then
	$$\sum^\infty_{n=0} a_n + \sum^\infty_{n=0} b_n = \sum^\infty_{n=0} (a_n + b_n)$$
	converges as well.
\end{property}

\begin{property}
	$\forall c$, if $\sum^\infty_{n=0} a_n$ converges, then
	$$c\sum^\infty_{n=0} a_n = \sum^\infty_{n=0} c(a_n)$$
	Also converges
\end{property}

\begin{definition}
	The \textbf{Cauchy Criterion} states that $\sum^\infty_{n=0} a_n$ converges iff $\forall \epsilon > 0, \exists N$ such that $m>n > N$
	$$|S_m - S_n|=|\sum^m_{k=n+1} a_k|<\epsilon$$
\end{definition}

\begin{example}
	Apply Cauchy Criterion to $\sum^\infty_{n=0} \frac{1}{n^2}$ and $\sum^\infty_{n=0} \frac{1}{n(n+1)}$
\end{example}

To prove they converge, we first need the following proposition:

\begin{proposition}
	Suppose $\sum^\infty_{n=0} a_n$ converges, then $a_n \to 0$ as $n\to\infty$
\end{proposition}

\begin{proof} (Using the Cauchy Criterion)\\
Given $\epsilon > 0$, there exists $N$ such that for $m > n > N, |\sum^m_{k=n+1} a_k|<\epsilon$, take $m=n+1$, this implies that $|a_{n+1}|<\epsilon$ for all $n > N$, this then implies that $a_n \to 0$
\end{proof}

\begin{proposition}
	Suppose $a_n \geq 0$ then $\sum^\infty_{n=0} a_n$ converges iff $\{ S_n \}$ is bounded above.
\end{proposition}

\begin{proof}
	Since $a_n \geq 0, \forall n$, then $\{ S_n \}$ is increasing. Therefore $\lim_{N\to\infty} S_N$ exists iff $\{ S_n \}$ is bounded above.
\end{proof}

\begin{example}
	$$\sum^\infty_{n=0} \frac{1}{n}, S_N = 1 +... + \frac{1}{N} > log(N), log(N)\to\infty \text{ as } n\to\infty$$
\end{example}

\subsubsection{Convergence Tests}

\begin{definition}
	\underline{The Integral Test:} let $f$ be a function defined for $x \geq 1$ and is integrable. Let $a_n = f(n), \forall n \in\mathbb{N}$ then $sum^\infty_{n=0} a_n$ converges iff
$$\int^\infty_1 f(x) dx = \lim_{y\to\infty} \int^y_1 f(x) dx$$ exists
\end{definition}

\begin{example}
	Consider $f(x)=x^a, a\in\mathbb{R}, a\neq -1$
	$$\int^y_1 x^a dx = \left.\frac{x^{a+1}}{a+1}\right|^y_1 = \frac{x^{y+1}}{y+1} - \frac{1}{a+1}$$
	So now we know that $\int^y_1 x^a dx$ exists iff $a < -1$, so equivalently:
	$$\sum^\infty_{a=1} \frac{1}{n^a} \text{ converges iff } a > 1$$
\end{example}














\end{document}



























