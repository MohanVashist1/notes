\documentclass[12pt]{article}

\usepackage{upgreek}

\usepackage{amsmath}

\usepackage{graphicx}
\graphicspath{ {imgs/} }

\usepackage{dsfont}

\usepackage{mathtools}

\usepackage{hyperref}

\usepackage[utf8]{inputenc}

\usepackage{mathtools}

\usepackage{textcomp}

\usepackage[english]{babel}

\usepackage{tikz}

\usepackage{tcolorbox}

\usepackage{amsthm,amssymb}

\setlength{\parindent}{0cm}

\renewcommand\qedsymbol{$\blacksquare$}

\usepackage{fancyhdr}
 
\pagestyle{fancy}
\fancyhf{}
\fancyhead[LE,RO]{Computability and Complexity -- Winter 2020}
\fancyhead[RE,LO]{Joshua Concon}
\fancyfoot[CE,CO]{\leftmark}
\fancyfoot[LE,RO]{\thepage}


\begin{document}

\title{CSCC63: Computability and Complexity\\ Lecture Notes}
\date{University of Toronto Scarborough -- Winter 2020}
\author{Joshua Concon}
\maketitle
Instructor is Eric Corlett. If you find any problems in these notes, feel free to contact me at conconjoshua@gmail.com.

\tableofcontents

\pagebreak

\section{LEC 1: Monday, January 6, 2020}

In this course, we want to explore the limitations of computation that go beyond processing power.

Consider the following problems:

\begin{enumerate}
  \item Given $x,y \in \mathbb{Z}$, what is $x^2 + y^3$?
  \item Given $x,y,z \in \mathbb{Z}$, is $z = x^2 + y^3$ true?
  \item Given $x,z$, does there exist a $y$ such that $z = x^2 + y^3$?
\end{enumerate}

Note that the first problem is not a \textbf{decision problem}, because it's solution is not yes/no or true/false, but the other two are. This course will mostly revolve around decision problems.

\paragraph{Decision Problem} A problem whose answer is Yes or No.

Now, consider the following. We'll define $f(x,y,z)$ as a program that solves the second problem. So the second problem returns true iff this function returns true and vice versa.
\\
Now, we'd call $(x,y,z)$ the input to the program $f$, but we want to separate the problem from the program. So we'll call the $(x,y,z)$ an \textbf{instance} of the second problem.

\paragraph{Instance} the input of a problem

\paragraph{Yes-Instance} the input of a decision problem that returns a yes

\paragraph{No-Instance} the input of a decision problem that returns a no
\\
\\
If we wanted to work with graphs, we'd have to encode a graph in a computer as an adjacency matrix, which we can turn into a string (which is computer readable). If we encode a graph $G$, we refer to its encoded format as $\langle G \rangle$
\\
\\
Note that there are some things we cannot represent in a computer, like $\pi$

\paragraph{Algorithm} A finite sequence of logical steps that always terminates.

\end{document}