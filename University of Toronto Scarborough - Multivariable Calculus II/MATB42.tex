\documentclass[12pt]{article}

\usepackage{upgreek}

\usepackage{amsmath}

\usepackage{amsthm}

\usepackage{graphicx}
\graphicspath{ {imgs/} }

\usepackage{dsfont}

\usepackage{mathtools}

\usepackage{hyperref}

\usepackage[utf8]{inputenc}

\usepackage{mathtools}

\usepackage{textcomp}

\usepackage[english]{babel}

\usepackage{tikz}

\usepackage{tcolorbox}

\usepackage{amsthm,amssymb}

\setlength{\parindent}{0cm}

\renewcommand\qedsymbol{$\blacksquare$}

\usepackage{fancyhdr}
 
\pagestyle{fancy}
\fancyhf{}
\fancyhead[LE,RO]{Multivariable Calculus II -- Winter 2018}
\fancyhead[RE,LO]{Joshua Concon}
\fancyfoot[CE,CO]{\leftmark}
\fancyfoot[LE,RO]{\thepage}

\theoremstyle{plain}

\newtheorem*{remark}{Remark}
\newtheorem*{note}{Note}
\newtheorem{theorem}{Theorem}[section]

\theoremstyle{definition}

\newtheorem{definition}[theorem]{Definition}
\newtheorem{lemma}[theorem]{Lemma}
\newtheorem{example}[theorem]{Example}
\newtheorem{proposition}[theorem]{Proposition}
\newtheorem{corollary}[theorem]{Corollary}
\newtheorem{property}[theorem]{Property}

\begin{document}

\title{MATB42: Multivariable Calculus II\\ Lecture Notes}
\date{University of Toronto Scarborough -- Winter 2018}
\author{Joshua Concon}
\maketitle
Pre-reqs are MATB41. Instructor is Eric Moore. If you find any problems in these notes, feel free to contact me at conconjoshua@gmail.com.

\tableofcontents

\pagebreak

\section{Friday, January 5, 2018}

\subsection{Fourier Expansions}

In this section, we will focus on single variable calculus, (so where $f:\mathbb{R} \mapsto \mathbb{R}$)\\
\\
Let us say that we have a function $f(x)$ and we want to approximate it. We can use an $n$th degree Taylor Polynomial, but this requires that $f(x)$ has at least $n$ derivatives at some point $x_0$ and the $k$th derivative of $f$ ($f^{(k)} (x)$) is determined by properties of $f$ in some neighbourhood of $x_0$, but what about outside this neighbourhood? How can we be certain of the approximation outside of this neighbour.\\
\\
Our problem here is that Taylor Polynomial may only approximate "near" $x_0$\\
\\
Now, consider the following function:

$$\Delta (x) = \begin{cases}
1, &\lfloor x \rfloor < x, \lfloor x \rfloor\text{ is odd} \\
0, &\lfloor x \rfloor < x, \lfloor x \rfloor\text{ is even}\\
\end{cases}
$$

In this function, Tayler returns either 0 or 1 depending on your choice of $x_0$ and cannot work for an $x_0 = p \in \mathbb{Z}$. Therefore Taylor polynomials cannot reflect the true nature of this function. Taylor provides a "local" approximation, but we want a "global" approximation. We need an approximation that is more precise over an interval at the cost of being not as precise as precise at any particular $x_0$.\\
\\
Note that the example function is \textbf{periodic}.\\
\\

\begin{definition}
	A function $y=f(x)$ such that $f(x)=f(x+p), p \neq 0, \forall x$ is said to be \textbf{periodic} of period $p$
\end{definition}

\begin{example}
	The periodic function $\Delta (x)$ is of period 2.
\end{example}

What we want is a global approximation of a periodic function, and the Fourier Approximation will be periodic, so we can use it for exactly that.

\begin{definition}
	A \textbf{trigonometric polynomial of degree $N$} is an expression of the form
	$$\frac{a_0}{2}+ \sum^N_{k-1} a_k cos(kx) + b_k sin(kx)$$ where the $a_i, b_i$ are constants.
\end{definition}

We know that $sin(x)$ and $cos(x)$ are the simplest periodic functions and repeat in intervals of $2\pi$, so $cos(kx)$ and $sin(kx)$ have period $\frac{2\pi}{k}$, but the smallest shared period is $2\pi$. If a trigonometric polynomial has period $2\pi$ and $f(x)$ has period $p$, then we must set $x=\frac{pt}{2\pi}$ to fix the period (where $t$ is a variable).\\
\\
So to approximate $y=f(x)$ by $F_N (x)$ for some $N$, we use the following equation:

$$F_N (x) = \frac{a_0}{2}+ \sum^N_{k-1} a_k cos(kx) + b_k sin(kx)$$

Now we need to choose the $a_k, b_k$. We can define it in the following way:

$$a_0 = \frac{1}{\pi} \int^{\pi}_{-\pi} f(x) dx$$
$$a_k = \frac{1}{\pi} \int^{\pi}_{-\pi} f(x) cos(kx) dx, k=1,2,3,...$$
$$b_k = \frac{1}{\pi} \int^{\pi}_{-\pi} f(x) sin(kx) dx, k=1,2,3,...$$

When defined in this way, $a_i, b_i$ are called the \textbf{Fourier Coefficients} of $f$ over the interval $[-\pi, \pi]$ and we call $F_N (x)$ the \textbf{Fourier Polynomial of degree $N$}.\\
\\
So why do we add the $\frac{a_0}{2}$? It is the average value of $f$ over $[-\pi, \pi]$.

\begin{note}
	sometimes you will see $a_0$ used instead of $\frac{a_0}{2}$ in the Fourier polynomial where $$a_0 = \frac{1}{2\pi} \int^{\pi}_{-\pi} f(x) dx$$
\end{note}

\begin{example}
	Consider $f(x)=\frac{-x}{2}$ over $[-\pi, \pi]$. Use Fourier Approximation.\\
	\\
	$$a_k = \frac{1}{\pi} \int^{\pi}_{-\pi} f(x) cos(kx) dx = \frac{1}{\pi} \int^{\pi}_{-\pi} (-\frac{x}{2}) cos(kx) dx \overset{odd}{=} 0$$
	$$a_0 = \frac{1}{\pi} \int^{\pi}_{-\pi} f(x) dx = \frac{1}{\pi} \int^{\pi}_{-\pi} (-\frac{x}{2}) dx \overset{odd}{=} 0$$
	\begin{align*}
		b_k &= \frac{1}{\pi} \int^{\pi}_{-\pi} f(x) sin(kx) dx\\
		&= \frac{1}{\pi} \int^{\pi}_{-\pi} (-\frac{x}{2}) sin(kx) dx\\
		&\overset{even}{=} - \frac{1}{\pi} \int^{\pi}_{-\pi}  x sin(kx) dx\\
		&\underset{u=x, dv=sin(kx)dx}{\overset{even}{=}} - \frac{1}{\pi} [-\frac{1}{k} x cos(kx) + \frac{1}{k^2} sin(kx)]^\pi_0\\
		&= \frac{1}{\pi k} [\pi cos(k\pi)]\\
		&= \frac{1}{k} cos(k\pi)\\
		&= \frac{(-1)^k}{k}
	\end{align*}
	
	Thus we have:
	$$F_N (x) = -sin(x) + \frac{1}{2}sin(2x) - \frac{1}{3}sin(3x) + \frac{1}{4}sin(4x) + ...$$
	$$F_1 (x) = -sin(x)$$
	$$F_2 (x) = -sin(x) + \frac{1}{2}sin(2x)$$
	$$F_3 (x) = -sin(x) + \frac{1}{2}sin(2x) - \frac{1}{3}sin(3x)$$
	$$...$$
	And so on.

\end{example}

\newpage

\section{Monday, January 8, 2017}

continuing from the last lecture...

\begin{example}
    (continued from example 1.4)\\
    $f(x) = \frac{-x}{2}$\\
    $F_N (x) = -sin(x) + \frac{1}{2}sin(2x) - \frac{1}{3}sin(3x) + \frac{1}{4}sin(4x) + ...$\\
    \\
    This can be extended to a Fourier Series:
    $$F_N (x) = \sum^\infty_{k=1} (-1)^k \frac{sin(kx)}{k}$$
\end{example}

\begin{definition}
    For $f:\mathbb{R}\mapsto\mathbb{R}$, the Fourier Series for $f$ is
    $$F(x) = \frac{a_0}{2}+ \sum^\infty_{k=1} a_k cos(kx) + b_k sin(kx)$$
    where $a_i, b_i$ are Fourier coefficients.
\end{definition}

The $N$th degree Fourier Polynomial can be regarded as the $N$th partial sum of the series.\\
\\
We haven't talked about convergence yet, but for now, we will assume the series converges $(f(x)=F(x))$

\begin{definition}
    Function $a_k cos(kx) + b_k sin(kx)$ is the $k$th harmonic of $f$. The Fourier Series expresses $f$ in terms of its harmonics.
\end{definition}

\begin{note}
    (Looking at Harmonics in a Musical Sense):\\
    the $1$st harmonic is the fundamental harmonic of $f$ (the fundamental tone).\\
    The $2$nd harmonic is the first overtone.
\end{note}

(completely rewrite this amplitude section)

\begin{definition}
    The amplitude of the $k$th harmonic is 
    $$A_k = \sqrt{(a_k)^2 + (b_k)^2}$$
    And note that
    $$a_k = A_k sin\alpha, b_k = A_k cos\alpha$$
\end{definition}

\begin{definition}
    The energy $E$ of a periodic function $f$ of period $2\pi$ is
    $$E = \frac{1}{\pi} \int^\pi_{-\pi} [f(x)]^2 dx$$
\end{definition}

So the energy of the $k$th harmonic is
$$E = \frac{1}{\pi} \int^\pi_{-\pi} [a_k cos(kx) + b_k sin(kx)]^2 dx = (a_k)^2 + (b_k)^2 = (A_k)^2$$

And the energy of the constant term is

$$\frac{1}{\pi} \int^\pi_{-\pi} [a_0]^2 dx = 2(a_0)^2$$
So we put $A_0 = \frac{1}{\sqrt{2}} a_0$.\\
\\
For a "nice" periodic function, we have the following equation:

$$E = A_0^2 + A_1^2 + A_2^2 + ...$$
This is known as the Energy Theorem, and comes from the study of periodic waves.\\
\\
We can draw a graph of this as $A_k^2$ against $k$ (This graph is known as the Energy Spectrum of $f$). It shows how the energy of $f$ is distributed among its harmonics.\\
\\
\begin{note}
    Notice that
    $$E = \frac{1}{\pi} \int^\pi_{-\pi} [f(x)]^2 dx = \frac{a_0^2}{2}+ \sum^\infty_{k=1} (a_k^2 + b_k^2) \text{ Parseval's Equation}$$
\end{note}

Assume a function $f$ of period $2\pi$ is the sum of a trigonometric series

$$f(x) = \frac{a_0}{2}+ \sum^\infty_{k=1} (a_k cos(kx) + b_k sin(kx)) \text{ on the interval } [-\pi,\pi]$$

Multiply by $cos(mx)$ and integrate to get
\begin{align*}
    \int^\pi_{-\pi} f(x)cos(mx) dx &= \frac{a_0^2}{2} \int^\pi_{-\pi} cos(mx) dx + \int^\pi_{-\pi} [\sum^\infty_{k=1} (a_k cos(kx) + b_k sin(kx))]dx\\
    &= \frac{a_0^2}{2} \int^\pi_{-\pi} cos(mx) dx +  \sum^\infty_{k=1} (a_k \int^\pi_{-\pi}cos(kx)dx + b_k \int^\pi_{-\pi}sin(kx)dx)
\end{align*}

\begin{note}
    Recall the following trigonometric identities:
    \begin{enumerate}
        \item $cosAcosB = \frac{1}{2} [cos(A+B) + cos(A-B)]$
        \item $cosAsinB = \frac{1}{2} [sin(A+B) + sin(A-B)]$
        \item $sinAsinB = \frac{1}{2} [cos(A-B) - cos(A+B)]$
    \end{enumerate}
\end{note}

\newpage

\section{Friday, January 12, 2018}

Continuing from where we left off.

\begin{align*}
    \int^\pi_{-\pi} f(x)cos(mx) dx &= \frac{a_0^2}{2} \int^\pi_{-\pi} cos(mx) dx + \int^\pi_{-\pi} [\sum^\infty_{k=1} (a_k cos(kx) + b_k sin(kx))]dx\\
    &= \frac{a_0^2}{2} \int^\pi_{-\pi} cos(mx) dx +  \sum^\infty_{k=1} (a_k \int^\pi_{-\pi}cos(kx)dx + b_k \int^\pi_{-\pi}sin(kx)dx)
\end{align*}

We know the following from trigonometric identities
$$\int^\pi_{-\pi} cos(kx)cos(mx) dx = \begin{cases}
    0, &k\neq m\\
    \pi, &k=m
\end{cases}$$
As well as the following from odd function properties
$$\int^\pi_{-\pi} cos(kx) dx = 0$$
$$\int^\pi_{-\pi} sin(kx)cos(mx) dx = 0$$
So now we get
\begin{align*}
    \int^\pi_{-\pi} f(x)cos(mx) dx
    &=  \frac{a_0^2}{2} \int^\pi_{-\pi} cos(mx) dx +  \sum^\infty_{k=1} (a_k \int^\pi_{-\pi}cos(kx)dx + b_k \int^\pi_{-\pi}sin(kx)dx)\\
    &= a_m \pi\\
    &\implies a_m = \frac{1}{\pi} \int^\pi_{-\pi} f(x)cos(mx) dx
\end{align*}

\begin{example}
    Lets take
    $$f(x) = \begin{cases}
    1, &0\leq x < \pi\\
    -1, &-\pi \leq x < 0
    \end{cases}$$
    
    Note that this is an odd function, therefore $a_k = 0, \forall k \geq 0$. So now lets calculate $b_k$.
    
    \begin{align*}
        b_k &= \frac{1}{\pi} \int^\pi_{-\pi} f(x)sin(kx) dx\\
        &\overset{even}{=} \frac{2}{\pi} \int^\pi_{0} f(x)sin(kx) dx\\
        &= \frac{2}{\pi} \int^\pi_{0} sin(kx) dx\\
        &= \frac{2}{k\pi} [ -cos(kx) ]^\pi_0\\
        &= \begin{cases}
            \frac{4}{k\pi}, &\text{$k$ is odd}\\
            0, &\text{$k$ is even}
        \end{cases}
    \end{align*}
    
    And now lets right out the Fourier Polynomial $(F_N (x))$\\
    \underline{if $N$ is odd:}
    $$F_N (x) = \frac{4}{\pi}sin(x) + \frac{4}{3\pi}sin(3x) + \frac{4}{5\pi}sin(5x) + ...$$
    \underline{if $N$ is even:}
    $$F_N (x) = F_{N-1} (x)$$
    We can also write it as a Fourier Series
    $$F(x) = \frac{4}{\pi} \sum^\infty_{l=0} \frac{sin((2l+1)x)}{2l+1}$$
    
    The energy of the function is
    $$E = \frac{1}{\pi} \int^\pi_{-\pi} [f(x)]^2 dx = \frac{2}{\pi} \int^\pi_0 dx = 2$$
    
    The amplitutde of the $k$th harmonic is
    $$A_k = \sqrt{a_k^2 + b_k^2} = \sqrt{0 + \frac{16}{k^2\pi^2}} = \frac{4}{k\pi}$$
    
    The energy of the $k$th harmonic is
    $$A_k^2 = \frac{16}{k^2 \pi^2}$$
    
    Note that for this example, both the energy and the amplitude are 0 at an even $k$.\\
    \\
    Lets now evaluate the energy spectrum:
    $$k=1, E= \frac{16}{\pi^2} \approx 1.62, \frac{1.62}{2}=0.81=81\%$$
    $$k=3, E= \frac{16}{9\pi^2} \approx 0.18, \frac{0.18}{2}=0.09=9\%$$
    $$k=5, E= \frac{16}{25\pi^2} \approx 0.06, \frac{0.06}{2}=0.03=3\%$$
    $$k=7, E= \frac{16}{49\pi^2} \approx 0.03, \frac{0.03}{2}=0.015=1.5\%$$
    
\end{example}

However, we do not need to exclusively work with the interval $[-\pi, \pi]$ we can even work over any interval of length $2\pi$.

\subsection{General Fourier Series (interval of length $2\pi$)}

$$a_k = \frac{1}{\pi} \int^{c+2\pi}_{c} f(x) cos(kx) dx, k=1,2,3,...$$
$$b_k = \frac{1}{\pi} \int^{c+2\pi}_{c} f(x) sin(kx) dx, k=1,2,3,...$$

What about for $f$ if $f$ has period $p$?

$$f(x+p) = f(x), \forall x, \exists p \neq 0$$

We then substitute $x = \frac{pt}{2\pi}$ which gives a new function $f_p (t) = f(\frac{pt}{2\pi})$ with period $2\pi$. So

$$f_p (t+2\pi) = f(\frac{p}{2\pi} (t+2\pi)) = f(\frac{pt}{2\pi} + p) = f(\frac{pt}{2\pi}) = f_p (t)$$

So how about the Fourier Expansion for $f_p (t)$? To find this, we must replace $t$ by $\frac{2\pi x}{p}$ giving for $f(x)$.

$$F(x) = \frac{a_0}{2}+ \sum^\infty_{k-1} a_k cos(\frac{2nx\pi}{p}) + b_k sin(\frac{2nx\pi}{p})$$
$$a_k = \frac{2}{p} \int^{c+p}_{c} f(x) cos(\frac{2nx\pi}{p}x) dx, k=1,2,3,...$$
$$b_k = \frac{2}{p} \int^{c+p}_{c} f(x) sin(\frac{2nx\pi}{p}x) dx, k=1,2,3,...$$

For any function defined on $[a,b]$, we can extend $f$ to all of $\mathbb{R}$ as a periodic function. Given a periodic function $f_E$ from $f$ of period $p=b-a$, we now have:

$$a_k = \frac{2}{b-a} \int^{b}_{a} f(x) cos(\frac{2nx\pi}{b-a}x) dx, k=1,2,3,...$$
$$b_k = \frac{2}{b-a} \int^{b}_{a} f(x) sin(\frac{2nx\pi}{b-a}x) dx, k=1,2,3,...$$

\begin{example}
    Take the function $f(x) = x, 0\leq x < 1$ and extend it with period 1. For $k\neq 0$:
    \begin{align*}
        a_k &= 2 \int^{1}_{0} xcos(2k\pi x) dx\\
        &\overset{parts}{\underset{u=v, dv=cos(2\pi kx)dx}{=}} 2[\frac{xsin(2k\pi x)}{2\pi k}]^1_0 - \frac{2}{2k \pi} \int^1_0 sin(2k\pi x) dx\\
        &=0
    \end{align*}
    $$a_0 = 2 \int^1_0 x dx = [x^2 ]^1_0$$
    \begin{align*}
        b_k &= 2\int^1_0 x sin(2k\pi x)dx\\
        &\underset{u=x, dv=sin(2k\pi x)dx}{\overset{parts}{=}} 2[\frac{-xcos(2k\pi x)}{2\pi k}]^1_0 + \frac{1}{k\pi} \int^1_0 cos(2k\pi x)dx\\
        &= \frac{-cos(2k\pi)}{k\pi}\\
        &= \frac{-1}{k\pi}
    \end{align*}
    So the Fourier Series will be
    $$F(x) = \frac{1}{2} - \frac{1}{\pi} [sin(2\pi x) + \frac{sin(4\pi x)}{2} + \frac{sin(6\pi x)}{3} + ...]$$
\end{example}

\begin{example}
    $f(x) = |x|, -\pi < x \leq \pi$. Since $f(x)$ is even, $b_k=0, \forall k\in\mathbb{N}$.
    
    $$a_0 = \frac{1}{\pi} \int^\pi_{-\pi} |x|dx \overset{even}{=} \frac{2}{\pi} \int^\pi_0 x dx= \pi$$
    
   \begin{align*}
   	a_k &= \frac{1}{\pi} \int^\pi_{-\pi} |x| cos(kx) dx\\
	&\overset{even}{=} \frac{2}{pi} \int^\pi_0 x cos(kx) dx\\
	&\underset{u=x, dv=cos(kx)dx}{\overset{parts}{=}} \frac{2}{\pi} [\frac{x sin(kx)}{k} + \frac{cos(kx)}{k^2}]^\pi_0\\
	&= \frac{2}{\pi k^2} (cos(k\pi) - 1)\\
	&=\begin{cases}
		0, &\text{$k$ is even}\\
		\frac{-4}{\pi k^2}, &\text{$k$ is odd}
	\end{cases}
   \end{align*}
   
   So we end up with the Fourier Series
   
   $$F(x) = \frac{\pi}{2} - \frac{4}{\pi} \sum^\infty_{l=0} \frac{cos((2l+1)x)}{(2l+1)^2}$$

\end{example}

\begin{example}
	$f(x) = x, -\pi < x \leq \pi$. Note that since $f(x)$ is odd, $a_k = 0, \forall k \geq 0$
	\begin{align*}
		b_k &= \frac{1}{\pi} \int^\pi_{-\pi} x sin(kx) dx\\
		&\underset{u=x, dv=sin(kx)dx}{\overset{parts}{=}} \frac{1}{\pi} [\frac{-xcos(kx)}{k}]^\pi_{-\pi} + \frac{1}{\pi} \int^\pi_{-\pi} cos(kx) dx\\
		&= \frac{1}{\pi} [\frac{-\pi cos(k\pi)}{k} + \frac{-\pi cos(-k\pi)}{k}]\\
		&= \frac{-2}{k} cos(k\pi)\\
		&= \frac{(-1)^{k+1} 2}{k}
	\end{align*}
	And the Fourier Series of this is
	$$F(x) = \sum^\infty_{k=1} (-1)^{k+1} \frac{2}{k} sin(kx)$$
\end{example}

To get a Fourier cosine series or a Fourier sine series, we need
\begin{itemize}
	\item{an $f$ defined on the interval $[0,a]$, and we must extend this interval to also include $[-a,0)$ to give an even or odd function on $[-a,a]$}
	\item{$f(-t)=f(t), -a\leq t < 0$ for the even extension}
	item{$f(-t)=-f(t), -a\leq t < 0$ for the odd extension}
\end{itemize}

\end{document}



























