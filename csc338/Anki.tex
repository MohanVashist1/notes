\documentclass[12pt]{article}

\usepackage{upgreek}

\usepackage{amsmath}

\usepackage{graphicx}
\graphicspath{ {imgs/} }

\usepackage{dsfont}

\usepackage{mathtools}

\usepackage{hyperref}

\usepackage[utf8]{inputenc}

\usepackage{mathtools}

\usepackage{textcomp}

\usepackage[english]{babel}

\usepackage{tikz}

\usepackage{tcolorbox}

\usepackage{amsthm,amssymb}

\setlength{\parindent}{0cm}

\renewcommand\qedsymbol{$\blacksquare$}

\usepackage{fancyhdr}

\begin{document}

\title{CSC338 Anki}
\maketitle

\section{LEC 1}

\paragraph{Well-posed}

A problem is \_ if the solution
\begin{itemize}
    \item exists
    \item is unique
    \item depends continuously on problem data
\end{itemize}

\paragraph{Ill-posed}

not well-posed

\paragraph{Is the solution of $f(x) = x^2 - 1$ well-posed?}

Yes

\paragraph{Solving $0 = x^2 - 1$ is well-posed?}

No

\paragraph{Sensitivity}

Small changes in the input results in large changes in output

\paragraph{Can a solution be sensitive if a problem is well-posed?}

Yes, it may be

\paragraph{Accuracy}

the "closeness" of a computed solution to the actual solution

\paragraph{Absolute Error}

approx value − true value

\paragraph{Relative Error}

$\frac{\text{absolute error}}{\text{true value}}$

\paragraph{How do we derive Approx value from true value and relative error?}

Approx value = (true value)(1 + relative error)

\paragraph{$\hat{x}$}

Approximate value of $x$

\paragraph{$\hat{f}$}

Approximate function of $f$

\paragraph{$\hat{y}$}

Approximate of $y=f(x)$, so equal to $f(\hat{x})$

\paragraph{Forward Error}

$\Delta y = \hat{y} - y$

\paragraph{Backward Error}

$\Delta x = \hat{x} - x$

\paragraph{Computational Error}

$\hat{f}(\hat{x}) - f(\hat{x})$

\paragraph{Propogated data error}

$f(\hat{x}) - f(x)$

\paragraph{Total Error}

$\hat{f}(\hat{x}) - f(x) = \hat{f}(\hat{x}) - f(\hat{x}) + f(\hat{x}) - f(x)$

\paragraph{Truncation Error}

Difference between true result (for actual input) and result produced by given algorithm using exact arithmetic

\paragraph{Rounding Error}

Difference between result produced by given algorithm using exact arithmetic and result produced by same algorithm using limited precision arithmetic

\paragraph{well-conditioned (or insensitive)}

A problem is \_ if a relative change in the input causes a similar relative change in the solution

\paragraph{well-conditioned (or sensitive)}

A problem is \_ if a relative change in the input cases a much larger change in the solution

\paragraph{Conditioning Number $C_N$}

$C_N = \frac{|\Delta y/y|}{|\Delta x/x|} = \frac{|(f(\hat{x}) - f(x)) / f(x)|}{|(\hat{x} - x) / x|}$

\paragraph{What does it mean if $C_N > 1$}

Then the problem is ill-conditioned

\paragraph{What is the $C_N$ if $f$ is differentiable?}

$C_N = \frac{|(f(x + \Delta x) - f(x))| / |f(x)|}{|
    \Delta x|/ |x|} \approx \frac{|xf'(x)|}{f(x)}$
    
\paragraph{If $f$ is differentiable, is the definition good for when $f(x)=0$ (root finding)?}

No.

\paragraph{What is the intuition of the Condition Number?}

The condition number is the "amplification factor" relating relative forward error to relative backward error.

\paragraph{Relative Forward Error}

$\frac{\Delta y}{y}$

\paragraph{Relative Backward Error}

$\frac{\Delta x}{x}$

\paragraph{Stable}

An Algorithm is \_ if the result is relatively insensitive to perturbations during computation

\paragraph{Backward Error View}

An algorithm is stable if the result is the exact solution to a nearby problem

\paragraph{When can we obtain accurate solutions?}

When the problem is well conditioned and the algorithm is stable

\end{document}