\documentclass[12pt]{article}

\usepackage{upgreek}

\usepackage{amsmath}

\usepackage{graphicx}
\graphicspath{ {imgs/} }

\usepackage{tcolorbox}

\usepackage{enumitem}

\usepackage{listings}

\usepackage{mathtools}
\DeclarePairedDelimiter\ceil{\lceil}{\rceil}
\DeclarePairedDelimiter\floor{\lfloor}{\rfloor}

\usepackage{color}
 
\definecolor{codegreen}{rgb}{0,0.6,0}
\definecolor{codegray}{rgb}{0.5,0.5,0.5}
\definecolor{codepurple}{rgb}{0.58,0,0.82}
\definecolor{backcolour}{rgb}{0.95,0.95,0.92}
 
\lstdefinestyle{mystyle}{
    backgroundcolor=\color{backcolour},   
    commentstyle=\color{codegreen},
    keywordstyle=\color{magenta},
    numberstyle=\tiny\color{codegray},
    stringstyle=\color{codepurple},
    basicstyle=\footnotesize,
    breakatwhitespace=false,         
    breaklines=true,                 
    captionpos=b,                    
    keepspaces=true,                 
    numbers=left,                    
    numbersep=5pt,                  
    showspaces=false,                
    showstringspaces=false,
    showtabs=false,                  
    tabsize=2
}
 
\lstset{style=mystyle}

\usepackage{dsfont}

\usepackage{hyperref}

\newtcolorbox{mybox}[3][]
{
  colframe = #2!25,
  colback  = #2!10,
  coltitle = #2!20!black,  
  title    = #3,
  #1,
}


\usepackage[utf8]{inputenc}

\usepackage{mathtools}

\usepackage{textcomp}

\usepackage[english]{babel}

\usepackage{tikz}

\usepackage{amsthm,amssymb}

\setlength{\parindent}{0cm}

\renewcommand\qedsymbol{$\blacksquare$}

\usepackage{fancyhdr}
 
\pagestyle{fancy}
\fancyhf{}
\fancyhead[LE,RO]{Distributed Computing}
\fancyhead[RE,LO]{Joshua Concon}
\fancyfoot[CE,CO]{\leftmark}
\fancyfoot[LE,RO]{\thepage}


\begin{document}

\title{Calculus\\ Notes}
\date{MAT157 Self-Study}
\author{Joshua Concon}
\maketitle
Not entirely sure about pre-requisites. If you find any problems in these notes, feel free to contact me at conconjoshua@gmail.com.

\tableofcontents

\pagebreak

\section{Chapter 1 - Basic Properties of Numbers}

\underline{Some properties of Numbers}
\begin{itemize}
	\item{(Associative law for addition) $a + (b + c) = (a + b) + c$}
	\item{(Existence of an additive identity) $a + 0 = 0 + a = a$}
	\item{(Existence of additive inverses) $a + (-a) = (-a) + a = 0$}
	\item{(Commutative law for addition) $a + b = b + a$}
	\item{(Associative law for multiplication) $a\cdot(b\cdot c) = (a\cdot b)\cdot c$}
	\item{(Existence of a multiplicative identity) $a\cdot 1 = 1\cdot a = a; 1\neq 0$}
	\item{(Existence of a multiplicative inverses) $a \cdot a^{-1} = a^{-1} \cdot a = 1, \text{ for } a\neq 0$}
	\item{(Commutative law for multiplication) $a \cdot b = b \cdot a$}
	\item{(Distributive law) $a\cdot(b+c) = a\cdot b + a \cdot c$}
	\item{(Trichotomy law) For every number $a$, one and only one of the following holds}
	\begin{enumerate}
		\item{$a = 0$}
		\item{$a \in\mathbb{R}^{+}$}
		\item{$-a \in\mathbb{R}^{+}$}
	\end{enumerate}
	\item{(Closure under addition) If $a$ and $b$ are in $\mathbb{R}^{+}$, then $a+b$ is in $\mathbb{R}^{+}$}
	\item{(Closure under multiplication) If $a$ and $b$ are in $\mathbb{R}^{+}$, then $a\cdot b$ is in $\mathbb{R}^{+}$}
\end{itemize}

\begin{tcolorbox}[title=Absolute Value]
	$$|x| = \begin{cases} -x \text{ if } x < 0 & \\ x\text{ if } x \geq 0 \end{cases}$$
\end{tcolorbox}

\begin{mybox}{red}{Theorem}
	$$|a+b| \leq |a| + |b|$$
\end{mybox}

\newpage

\section{Chapter 2 - Number of Various Sorts}

\subsection{Simple Induction}

Mathematic induction states that a statement $P(x)$ is true for all natural numbers $x$ provided that

\begin{itemize}
	\item{$P(1)$ is true}
	\item{Whenever $P(k)$ is true, $P(k+1)$ is true}
\end{itemize}

\subsection{Complete Induction}

The properties of complete induction are a bit different, they just have different properties to prove that $A$ is in the set of all natural numbers, as shown
\begin{enumerate}
	\item{1 is in $A$}
	\item{$k+1$ is in $A$, if $1,...,k$ are in $A$}
\end{enumerate}

\subsection{Different Numbers}

From Chapter 1, we have all these properties, but not all of them work for the natural numbers.

 
\end{document}


























