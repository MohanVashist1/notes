\documentclass[12pt]{article}

\usepackage{upgreek}

\usepackage{amsmath}

\usepackage{graphicx}

\usepackage{dsfont}

\usepackage{hyperref}

\usepackage[utf8]{inputenc}

\usepackage{mathtools}

\usepackage{textcomp}

\usepackage[english]{babel}

\usepackage{tikz}

\usepackage{tcolorbox}

\usepackage{amsthm,amssymb}

\setlength{\parindent}{0cm}

\renewcommand\qedsymbol{$\blacksquare$}

\usepackage{fancyhdr}
 
\pagestyle{fancy}
\fancyhf{}
\fancyhead[LE,RO]{Introduction to Databases -- Fall 2017}
\fancyhead[RE,LO]{Joshua Concon}
\fancyfoot[CE,CO]{\leftmark}
\fancyfoot[LE,RO]{\thepage}


\begin{document}

\title{CSCC43: Introduction to Databases\\ Lecture Notes}
\date{University of Toronto Scarborough -- Fall 2017}
\author{Joshua Concon}
\maketitle
Pre-reqs are CSCB63 and STAB52.
Instructor is Dr. Marzieh Ahmadzadeh. Check RateMyProf. If you find any problems in these notes, feel free to contact me at conconjoshua@gmail.com.

\tableofcontents

\pagebreak

\section{Thursday, September 7, 2017}

\subsection{Definitions}

\paragraph{Data} Objects or events that could be recorded on a computer media. This includes stuff like an email, an address, a student identification number, etc.

However, there are two types of Data



\end{document}


























